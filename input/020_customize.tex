% koneo 2016-02-22
%
% Customize
% 
% Hier wird ein Teil der verwendeten Variabeln gesetzt. Die Variabeln hier enthalten Text. Die Layout-Variabeln werden in 010_Layout gesetzt.

%
% Ersten Seite 1_cover.tex
%
\newcommand \gsempro{Semesterprogramm}
\newcommand \gcsem{110. Couleursemester}
\newcommand \gorga{Christlich Deutsche Studentenverbindung}
\newcommand \gorgb{Nibelungen zu Siegen}
\newcommand \gorgc{im Wingolfsbund}
\newcommand \gsem{Sommersemester 2016}

%
% Zweite Seite 2_einladung.tex
%

\newcommand \gzweiteSeiteTitel {Einladung und Vorstellung}
	
\newcommand \gzweiteSeite {
	Die Aktivitas der Nibelungen zu Siegen gibt sich die Ehre, dieses Semesterprogramm ihren Kommilitonen, verehrten Gästen und lieben Bundesbrüdern zu überreichen. Es dient auch als herzliche und offizielle Einladung.\\
	\\
	Wir halten während des Studiums und darüber\, hinaus\,  gemeinschaftlich\, zusammen, unterstützen und fördern uns gegenseitig. Unsere Mitglieder sind in allen Fachbereichen zu finden. Wir bekennen uns aktiv zur freiheitlich-demokratischen\, Grundordnung\, Deutschlands und den christlichen Werten.\\
	\\
	Jeder ist eingeladen unser guter Gast zu sein. Auch Sie -- auch Du.
	\vfill
	
	Alle Veranstaltungen finden auf dem Verbindungshaus statt, falls nicht anders notiert. Für die Teilnahme an Kneipen bitten wir um frühzeitige Anmeldung.
	}

%
% Dritte Seite 3_beschreibung.tex
%

\newcommand \gdritteSeiteTitel {Konstante und Wingolfsbund}

\newcommand \gdritteSeite {
	Mittelpunkt unserer Aktivitäten ist unsere Konstante: Das geräumige \enquote{Haus der Nibelungen}, Sieghütter Hauptweg 110, mit 13 Studentenzimmern, Gesellschaftsraum und Festsaal. Dort leben, lernen und feiern wir zusammen, schließen und pflegen Freundschaften für das Leben. Professoren halten bei\, uns\, Vorträge, Absolventen berichten von ihren Berufserfahrungen, ältere Studenten helfen den Jüngeren und Freunde aus aller Welt schauen gerne vorbei.\\
	
	34 Verbindungen in deutschen, estnischen\,  und\, österreichischen Uni\-versitätsstädten bilden den Wingolfsbund. Wir pflegen gegenseitige Besuche, gemeinsame Veranstaltungen,  Fortbildungen und kehren ein auf Zwischenreise oder für Kurzurlaube . Die Gründ\-ung des Bundes im frühen 19. Jahrhundert dient dem Erhalt studentische Tradition, demokratische Ideale, christliche Werte und gemeinschaftliche Lebensweise.\\
	\\Diese Weltanschauung pflegen wir auch in Siegen.
}

%
% Vierte Seite 4_kontakt.tex
%

\newcommand \gvierteSeiteTitel {Kontaktdaten}

\newcommand \gvierteSeite {
	\textbf{Haus der Nibelungen}\\
	Sieghütter Hauptweg 110\\
	D-57072 Siegen \\
	+49 (0) 271 -- 770 218 6\\
	hallo@nibelungen-siegen.de\\
	www.nibelungen-siegen.de\\
	\\
	\textbf{Für die Aktivitas}\\
	Eric Schell \vulgo Ref\\
	+49 (0) 157 -- 861 870 16\\
	ref@nibelungen-siegen.de\\
	\\
	\textbf{Für den Alt-Herren-Verband}\\
	Detlef Roggenkamp \vulgo Robo\\
	robo@nibelungen-siegen.de\\
	\\
	\textbf{Für den BV 20}\\
	Volker Keßeler\\
	bv20@wingolf.org\\
	\\
	\textbf{Für den Wingolfsbund}\\
	Geschäftsführer Wilhelm Neusel\\
	gfdw@wingolf.org\\
	www.wingolf.org\\	
}
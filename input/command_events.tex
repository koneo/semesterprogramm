% Seite 4
	
%%%%%%%%%%%%%%%%%%%%%%%%%%%%%%%%%%%%%%%%%%%%%%%%%%%%%%%%%%%%%%%%%
% Programmpunkte hier
%%%%%%%%%%%%%%%%%%%%%%%%%%%%%%%%%%%%%%%%%%%%%%%%%%%%%%%%%%%%%%%%%

% Dieser Befehl fügt für einen Monat eine Veranstaltungsseite mit maximal 6 Veranstaltungen ein. 
% #1 Seitentitel, Monatsname
% #2 Veranstaltung 1
% #3 Veranstaltung 2
% #4 Veranstaltung 3
% #5 Veranstaltung 4
% #6 Veranstaltung 5
% #7 Veranstaltung 6

\newcommand \neuerMonat[1] {
	\newpage
	\pagestyle{plain}
	\pageTitle{#1}
	\fontsize{\defaultFontSize}{\defaultLineSpacing} \selectfont
}

% Dieser Befehl fügt eine Veranstaltung ein. Technisch werden hier entsprechende tabularx-Zeilen erzeugt. Zeile wird in \neuerMonat eingebaut.
% #1 Wochentag: {Mo, Di, Mi, Do, Fr, Sa, So} - ohne Punkt!
% #2 Datum des Tages: 01 - 31
% #3 Uhrzeit: {ab HH, HH Uhr, bis TT} - lose, strenge Zeitangabe, oder ganztätig bis TT.
% #4 Veranstaltungsart: {Kommers, Kneipe, Brunch, ...}
% #5 Veranstaltugnsmotto wird in kursiv geschrieben
% #6 Veranstaltungsort: {\adH} oder Kürzel. 
% #7 - leer. War mal: Veranstaltungsverantwortliche Status Bierspitz, zb. F Filius
\newcommand \neueVeranstaltung[7]{
	\begin{tabularx}{1\textwidth}{p{2.8cm} X}
	#1. #2. & \textbf{#4}, \emph{ #6 } \\
	#3      & \emph{{#5}} \\
            & \\ % #7)} \\
	\end{tabularx}
}

\newcommand \neueLeereVeranstaltung {}

\newcommand \neueAnkuendigung[3] {
	\null
	\vfill
	\begin{tabularx}{\textwidth}{||Xp{15cm}X||}
		\hhline{|t:===:t|}    
		&&\\  
		& \centering{#1}&\\ 
		&&\\  
		& \centering{\textbf{#2}}&\\  
		&&\\  
		& \centering{\emph{#3}}&\\
		&&\\    
		\hhline{|b:===:b|}
	\end{tabularx}	
}

\newcommand \neuerHinweis [2]{
	\null
	\vfill
	\begin{tabularx}{\textwidth}{||X p{15cm} X||}
		\hhline{|t:===:t|} 
		&&\\
		& \centering{\emph{#1}} &\\ 
		&&\\
		& \centering{\emph{\textbf{#2}}} &\\
		&&\\
		\hhline{|b:===:b|} 
	\end{tabularx}
}

%%% Seiten-Definition
% @param #1 : "blank-page" (keinen Header), "\defaultPageStyle" (mit Header)
% @param #2 : Zentrierte Seite? "\true" | "\false"
% @param #3 : Schriftgröße "\defaultFontSize" (16pt) | z.B. "12" (12pt)
% @param #4 : Zeilenabstand "\defaultLineSpacing" (24pt) | z.B. "18" (18pt)
% @param #5 : Titel der Seite
% @param #6 : Inhalt der Seite
\newcommand \addPage[6] {
	
	\pagestyle{#1} % blank-page (keinen Header), plain (mit Header)
	
	\pageTitle{#5} % Text im Header
	
	\ifthenelse{#2}{ % if (zentriert)
		\begin{center}
		}{} % else
		
		\fontsize{#3}{#4} \selectfont
		
		%%% Seiteninhalt
		%%
		%
		#6
		%
		%%
		%%%
		
		\ifthenelse{#2}{ % if (zentriert)
		\end{center}
	}{} % else
	
	%	\pagestyle{\noHeader}
	\newpage
	
} 

\newcommand \addEventPage[2] {
	
	\addPage
	{\defaultPageStyle}
	{\false}
	{\defaultFontSize}
	{\defaultLineSpacing}
	{#1}
	{#2}
	
}

\newcommand \eventTable[6] {
	
	\begin{tabularx}{1\textwidth}{p{1.8cm} p{2.2cm} X }	
		#1
		#2
		#3
		#4
		#5
		#6	
	\end{tabularx}
}

\newcommand \eventTableWithTwoCols[6] {
	
	\begin{tabularx}{1\textwidth}{p{4.35cm}	X }	
		#1
		#2
		#3
		#4
		#5
		#6	
	\end{tabularx}
}

\newcommand \eventRow[3] {
	#1 & #2	& #3\\
	& &\\
}

\newcommand \eventRowWithTwoCols[2] {
	#1 & #2\\
	&\\
}

\newcommand \emptyRow {
	& &\\
}

\newcommand \eventDamen {
	\includegraphics[width=0.74cm]{\neueVeranstaltungsStatusB}
}

\newcommand \eventAbendgardrobe{
	\includegraphics[width=0.74cm]{\neueVeranstaltungsStatusA}
}

\newcommand \eventDateCol[4] {
	{#1. #2.} %\ifthenelse{#3}{\eventDamen}{}\ifthenelse{#4}{\eventAbendgardrobe}{} 
}

\newcommand \eventIntervalCol[6] {
	{#1. #2. -- #3. #4}\ifthenelse{#5}{\eventDamen}{}\ifthenelse{#6}{\eventAbendgardrobe}{} 
}

\newcommand \eventTimeCol[3] {
	#1 #2 #3
}

\newcommand \eventDescriptionCol[4] {
	\textbf{#1} #2 \programmZitat{#3}{#4}
}

\newcommand \eventBox[3] {
	
	\begin{tabularx}{\textwidth}{||XcX||}
		\hhline{|t:===:t|}    
		&															&\\  
		& #1														&\\ 
		&															&\\  
		& \textbf{#2} 												&\\  
		&															&\\  
		& \emph{#3} 												&\\
		&															&\\    
		\hhline{|b:===:b|}
	\end{tabularx}
	
}

\newcommand \eventBoxBottom[3] {
	\null
	\vfill
	\begin{tabularx}{\textwidth}{||XcX||}
		\hhline{|t:===:t|}    
		&															&\\  
		& #1														&\\ 
		&															&\\  
		& \textbf{#2} 												&\\  
		&															&\\  
		& \emph{#3} 												&\\
		&															&\\    
		\hhline{|b:===:b|}
	\end{tabularx}
	
}
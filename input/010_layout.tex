% 2016-02-22 Koneo
% Hier werden Layoutparameter definiert, die für das ganze Dokument gelten sollen

%
%
\setlength {\voffset}       {26pt}
\setlength {\headheight}    {15pt}
\setlength {\headsep}       {4pt}
\setlength {\textheight}    {262pt}
\newcommand \glogo			{images/wappen.png}
\newcommand \gsgL			{21}
\newcommand \dfs            {10} 			% Default FontSize
\newcommand \dheadrulewidth {1pt}
\newcommand \dfsh			{12}
\newcommand \dls			{10} 			% Default LineSpace
\newcommand \ohneKopzeile   {blank-page} 	% "blank-page" (keinen Header),
\newcommand \mitKopfzeile   {plain} 		% (mit Header)
\newcommand \zentriert      {\boolean{true}}
\newcommand \linksbuending  {\boolean{false}}
\newcommand \ohneTitel      {}
\newcommand \vulgo			{$^{v}$/ }
\newcommand \gtw            {1.1cm}

\setlength{\parindent}{0pt}

\fancypagestyle{\ohneKopzeile}{
	\fancyhf{}
	\fancyhead{} 
	\fancyfoot{} 
	\renewcommand{\headrulewidth}{0pt}
}

\newcommand \seitenTitel[1]{
	\markboth{#1}{}
	\addcontentsline{toc}{section}{#1}
}

%%% Seiten-Definition
% #1: Kopfzeile
% #2: Zentriert oder linksbündig
% #3: Titel der Seite
% #4: Inhalt der Seite

\newcommand \neueSeite[4] {
	\newpage	
	\pagestyle{#1}							% Kopfzeile
	\seitenTitel{#3}						% Seitentitel
	\ifthenelse{#2}{\begin{center}}{}		% falls \zentiert,...
		\fontsize{\dfs}{\dls} \selectfont	% ... sonst
		#4									% Inhalt
		\ifthenelse{#2}{\end{center}}{}		% falls \zentriert,...
} 

\newcommand \neuerMonat[1] {
	\newpage
	\pagestyle{\mitKopfzeile}
	\seitenTitel{#1}
	\fontsize{\dfs}{\dls} \selectfont
}

% #1 Wochentag und Datum Mo. 14.
% #2 Uhrzeit
% #3 Ort
% #4 Art: Sieghüttenparty
\newcommand \veranstaltungA[4]{
	\begin{tabularx}{1\textwidth}{p{\gtw} X}
	#1 & \textbf{#4} #2			\\
								\\
	\end{tabularx}
}

% #1 Wochentag und Datum Mo. 14.
% #2 Uhrzeit
% #3 Ort
% #4 Art: Sieghüttenparty
% #5 Beschreibung
\newcommand \veranstaltungB[5]{
	\begin{tabularx}{1\textwidth}{p{\gtw} X}
	#1 	& \textbf{#4} #3			\\
	#2	& #5						\\
		&							\\
	\end{tabularx}
}


% #1 Wochentag und Datum Mo. 14.
% #2 Ort
% #3 Art: Akademischer Abend
% #4 Uhrzeit1
% #5 Beschreibung1
% #6 Uhrzeit2
% #7 Beschreibung2
\newcommand \veranstaltungC[7]{
	\begin{tabularx}{1\textwidth}{p{\gtw} X}
	#1      & \textbf{#3}	#2 	\\
	#4		& #5 				\\
    #6      & #7 				\\ 
            &   				\\
	\end{tabularx}
}

\newcommand \neueAnkuendigung[1] {
	\null
	\vfill
	\begin{tabularx}{\textwidth}{||p{1pt} X p{1pt}||}
		\hhline{|t:===:t|}    
		& 				& \\
		& \centering{#1}& \\
		& 				& \\ 
		\hhline{|b:===:b|}
	\end{tabularx}	
}

% % % % % % % % % % % % % % % % % % % % % % % % % % % % % % %
%
% Fancy

\fancypagestyle{plain}{
	\fancyhf{}
	\fancyhead{}
	\fancyhead[L]{{\fontsize{\dfsh}{\dls} \selectfont \textbf{\leftmark}}}
	\fancyfoot{}
	\renewcommand{\headrulewidth}{\dheadrulewidth}
}